% !TEX root = ../thesis.tex

\begin{summary}
本研究课题针对深度学习背景下的视频流处理任务提出了Aceel-Video Pipe(AVPipe)这一C++编程框架,为视频流处理任务提供开发流程与处理速度上的优化。\par
在开发流程上,我们通过对各类视频处理任务进行分析与总结,将整个视频处理任务抽象为数据包(StreamPacket),数据流(Stream),数据处理单元(PipeProcessor)这三者的互联与组合。为此,我们进行了C++相应代码的实现。在实现过程中,我们对处理流程可能涉及的数据类型、数据格式进行了统一与整合;对数据的多线程使用提供了必要的同步与阻塞机制;对多种深度学习推理引擎与计算硬件提供了支持,并简化和统一了调用接口。除了提供较完善的C++库的定义与实现,我们还进一步为便利开发提供了自动化工具。
通过定义处理流程的配置文件,AVpipe可实现自动化的C++代码生成,任务处理流程可视化以及处理流程运行时追踪与分析。\par

在处理速度上,我们为AVPipe提供了针对视频处理任务的多线程优化工具。通过视频处理任务运行时的性能分析,我们可以获得整个处理任务中每一模块的计算用时。多线程优化工具依据这些信息,利用视频处理流程存在的DAG计算模式与流水线计算模式, 将高耗时的瓶颈计算模块划分在不同的线程以实现数据处理的并行化计算。在多线程优化工具的具体实现上,我们提出了有针对视频处理任务特点的启发式DAG处理流程划分算法,在尽可能多线程平摊计算任务的同时,保证多阶段流水线的数据上下依赖关系。\par

最后,通过对一些具体视频处理任务的实现,我们展示了AVPipe的开发流程与性能优化。测试结果表明了我们提出的多线程优化策略的有效性,在某些情境下AVPipe的多线程优化可以提供将近一倍的速度提升。我们之后在GPU与VPU上的测试也进一步展现了AVPipe框架的通用性。\par

AVPipe将会作为一个开源项目在未来继续进行框架的优化与改进。我们会为其持续添加新的计算模块与开发文档;提供可视化的流程连接与参数配置的前端操作界面;移动平台的完善支持;更细粒度的针对视频处理任务的算法与系统层级的优化。我们十分期望AVPipe未来能够在实际生产开发中展现其应有的使用价值。
\end{summary}
