% !TEX root = ../thesis.tex

\chapter{背景和动机}
在这一节,我们会对深度学习背景下视频流处理的实际应用需求以及问题做总结。在此基础上,我们会列出对AVPipe设计的期望与目标。
\section{视频流处理的应用需求}\label{ch3:req}
随着机器学习,深度学习等人工智能技术的发展,不论在生产实践中还是在生活娱乐中,我们都期望人工智能技术能够给现有的应用带来创新。计算机视觉是应用人工智能最主要的领域之一,这之中对视频流的智能分析与推理可以说是许多智能视觉算法的主要实际应用形式。在生产实践中,无人驾驶、智能机器人等技术对智能视频流的推理分析有直接的依赖,智能工厂、智慧城市的建设也需要利用对视频的视觉分析,来实现自动化管理。在生活娱乐中,智能视频处理可以为用户完成内容创作提供便捷,比如近年来智能手机上兴起的AI视频增强功能、现实增强功能,如iPhone的Animoji\footnote{\url{https://support.apple.com/en-us/HT208190}}功能,在视频内容创作分享平台如抖音\footnote{\url{https://www.tiktok.com/}}、哔哩哔哩\footnote{\url{https://www.bilibili.com/}}等 %emmm 要不要脚注抖音 B站链接?
中提供的诸如智能抠图(图像分割),人物、姿态识别等的功能。高效,准确的视频处理也可以被应用在游戏创作中,为玩家带来玩法上的创新。\par
\section{视频流处理的现存问题}
针对\ref{ch3:req}中提到的新兴的对视频流处理的应用需求,我们也通过对相关产品的实际使用以及对一般视频流处理应用的开发实践,总结了现在视频流处理的应用与开发中存在的一些问题:
\begin{enumerate}
    \item 深度学习相关视频流处理应用的开发流程一般都是先利用Python环境下的各类机器学习框架(TensorFlow, PyTorch等)对算法验证与测试。
\end{enumerate}

\section{研究动机与目标}


\section{本章小结}