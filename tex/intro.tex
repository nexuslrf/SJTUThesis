% !TEX root = ../thesis.tex

\chapter{绪论}

\section{基于深度学习的计算机视觉技术}
以深度学习为代表的机器学习算法近年来在学术研究与生产应用的各个领域中备受关注。凭借其较强的特征与知识表征能力,飞速发展的深度学习算法逐渐在计算机视觉,自然语言处理,语音识别等识别与感知挑战中取得接近甚至超越人类的表现。%(加Cite呀)
深度学习模型往往依赖大规模数据集通过反向传播算法训练、优化得到的模型的权重参数,输入数据与这些权重参数进行的运算最终输出各类检测或识别的推理结果。无论是前向推理还是反向传播训练都需要处理存在大量的矩阵计算,因而深度学习的发展也离不开相关系统框架以及硬件架构的针对性优化与发展。\par
深度学习算法源于多层感知机模型(MLP)的不断改进与优化。MLP是一由全连接层(权重矩阵)和非线性激活函数共同构成的浅层神经网络。随着相关理论以及计算机算力的发展,神经网络的层数逐渐变深,不同类型的网络连接层、网络结构相继被提出。这一系列的发展使得深度学习模型有了更强的信息提取能力,在性能上逐步超越传统的人工设计的专家系统。\par
在计算机视觉领域,深度学习模型主要以卷积神经网络(CNN)的形式的存在。CNN借鉴了传统数字图像处理中卷积操作的概念,它权重共享以及平移不变的特性一方面大大减少了深层神经网络的参数,另一方面体现了CNN算法对不同图像特征提取的普遍适应性。
% emmm 要不要加个公式意思一下呢。。。 算了。。。
LeNet是CNN的早期代表模型,它被成功应用到了手写数字识别任务上。2012年的AlexNet使用了在GPU上实现的深层卷积神经网络结构,在当年的大规模视觉识别挑战(ImageNet)中取得了突破性的精度提升。AlexNet的出色表现也直接催生了后续深度学习技术的井喷式发展,从VGG,Inception,到ResNet,再到如今的网络架构搜索(NAS),CNN模型在精度和性能上都在不断地提升。如今各类视觉任务往往会利用这些出色的网络架构作为骨架来提取图像中的特征知识。\par
在具体视觉应用领域,计算机视觉也在近年来做到了从最初图像分类,人脸识别到目标检测,关键点检测,图像分割,图像生成等应用的全面发展。以目标检测为例,最开始R-CNN提出了区域推荐(Region Proposal)的方式完成目标检测。后来又有如Fast R-CNN,Faster R-CNN,Mask RCNN等算法将以区域推荐为主的检测方式逐步优化,不断提高模型的精度与运行效率。同时也有以YOLO系列为代表一系列检测算法通过对检测任务直接进行端对端的神经网络训练,模型的精简使得这种方法可以实现目标检测的实时推理。这些日益成熟的深度学习计算机视觉视觉也为无人驾驶,智能机器人,智慧城市等新兴研究方向提供了必要的技术支持,会在以后的生产生活中发挥更重要的作用。\par
如今深度学习算法在各个领域中已取得了初步的成效,当前深度学习技术发展主要关注在模型的压缩、剪枝、量化方法,自动化的机器学习算法开发(AutoML),深度神经网络黑盒模型的可解释性与安全性分析等方面。


\section{视频流处理模型与流程}

\section{视频流处理的优化策略}

\section{论文的主要内容与章节安排}





