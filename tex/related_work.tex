% !TEX root = ../thesis.tex

\chapter{相关工作}\label{related_work}
针对深度学习相关的视频处理优化属于在近年来兴起的研究课题,针对性的相关工作并不是很多。本章对一些较出名的相关项目进行介绍,同时对本框架中所涉及的技术与优化相关算法的现有工作做一简单介绍。

\section{视频流处理框架}
% 介绍 DataFlow, GStream,,Graph-API, MediaPipe, videoFlow
首先,GStreamer\cite{gstreamer}是一个以计算图的形式来创建有关流媒体的处理流程的开源框架,GStreamer最初于2001年发布,它支持多种流水线的构建,并且包含丰富的音、视频处理模块。但这个框架主要用于更为底层的音视频信号编辑与整合,而非本文所关注的视频内容的智能分析处理。\par
Dataflow\cite{akidau2015dataflow}是在2015年推出的数据流分析框架,它被应用在Google Cloud上提供云端数据流的分析与推理能力。该框架提供了对数据流处理的计算关系图的定义,但是该框架是应用在云端集群上并且以批量数据块为基本对象的模型。这并不适用于与构建用户端本地视频流处理部署,以数据流为处理对象的应用场景。\par 
OpenCV在其4.0版本中引入了Graph API\cite{matveev2018opencv},这一API可以使用户通过计算图的形式定义一系列对图像
的处理流程,然而Graph API的局限性在于它处理流程是基于OpenCV中Mat数据封装来进行的,因此处理流程需要依赖OpenCV对各类图像处理算法的支持,同时Mat的数据封装对神经网络推理所需要用到的高维张量的表示能力有限,无法完整定义在整个视频检测分析处理流程。\par
Videoflow\cite{deArmas2019videoflow}是一个于2019年开源的Python的视频流处理框架。它提供了对视频流处理完整步骤的简单的图定义,帮助开发人员快速完成视频处理算法的原型开发。Videoflow仅对处理流程图提供简单的数据同步管理,缺少针对性优化。此外Python语言的使用也限制了该框架的进一步性能优化以及实际部署应用。\par
MediaPipe\cite{lugaresi2019mediapipe}是Google在2019年CVPR上公开的数据流处理机器学习应用开发框架。
它用定义计算图的方式,构建使用了多种形式时间序列数据(如视频、音频以及传感器数据)进行机器学习分析推理流程的框架。针对流数据处理任务的快速部署与应用,MediaPipe使用TensorFlow\cite{abadi2016tensorflow}做为其深度学习推理引擎,提供多平台的支持与加速,包括各大桌面与移动平台。相比之前的数据流出框架,MediaPipe提供更为灵活的数据传输管道以及计算图定义。MediaPipe使用线程池模型将处于计算图中的不同阶段的数据包进行动态调度到不同的线程中运行。MediaPipe框架确实为流媒体分析处理应用的创建与部署提供了很大的帮助,但MediaPipe项目也有其局限性。首先,和其他很多Google开源项目一样,MediaPipe使用了很多Google自有的代码库与模块进行构建,自定义MediaPipe的计算模块对普通用户来说有较高的学习成本;其次,MediaPipe目前仅支持通过TensorFlow创建的视频处理任务,这对也使得该框架无法利用一些针对性优化的深度学习推理引擎(见\ref{related:dl_engine})来提升性能;
另外,MediaPipe的线程池模型为保证数据的依赖与同步有复杂的控制逻辑,在某些情况\footnote{如数据队列中多数数据包关联的线程需要运行高耗时的网络推理模块,但计算资源不能支持同时进行多个网络推理,从而造成多数线程的阻塞。}下需要用户手动调整不同线程的计算范围。\par
本研究所提出的Accel-Video Pipe对以上项目的优点都有所借鉴。MediaPipe作为如今最受关注的相关项目,目前仍在处于积极开发阶段。我们期望AV Pipe框架能够有MediaPipe相当的可用性,并对MediaPipe所存在的局限进行优化与改进。\par
\section{深度学习推理引擎}\label{related:dl_engine}
% @TODO 在这里加一个运行速度对比的表格!

\section{DAG任务的调度与划分}
% 多阶段有依赖任务的调度与划分

\section{本章小结}
